\documentclass{beamer} 
\usetheme{CambridgeUS}
\usepackage{enumerate}


\usepackage{color} 

\theoremstyle{definition} 
\newtheorem{defi}{Definition}  
\newtheorem{thm}{Theorem}
\newtheorem{lem}{Lemma}
\newtheorem{quest}{Question}
\newtheorem{conj}{Conjecture}
\newtheorem{ex}{Example}

\title[Geometric measure via singular traces]{Geometric measure on quasi-Fuchsian groups\\ via singular traces. Part II.}
\author[Dmitriy Zanin]{Dmitriy Zanin\\
(in collaboration with A. Connes and F. Sukochev)}
\institute[]{University of New South Wales}

\begin{document}

\begin{frame} 
\titlepage
\end{frame}

\section{Part (a) of the main result}

\begin{frame}
\frametitle{What exactly to prove?}

We aim to prove $[F,M_Z]\in\mathcal{L}_{p,\infty},$ where $p$ is the Hausdorff dimension of $\Lambda(G).$

By Peller theorem, it suffices to show that
$$k\in L_{p,\infty}(\mathbb{D},(1-|z|^2)^{-2}dzd\bar{z}).$$
Here,
$$k(z)=Z'(z)(1-|z|^2),\quad |z|<1.$$


\end{frame}

\begin{frame}
\frametitle{Fuchsian group conjugate to $G$}

Consider $G$ acting on $\Lambda(G)_{{\rm int}}.$ Let $\pi$ be the action of $G$ on the unit disk by the
formula
$$g\circ Z=Z\circ\pi(g),\quad g\in G.$$
Every $\pi(g)$ is a conformal automorphism of the unit disk; hence, $\pi(g)$ is fractional linear.

Thus, $\pi(G)$ is a group of fractional linear transformations preserving the unit circle, i.e. a Fuchsian group and it’s limit set is the unit circle $\mathbb{T},$ thus it is Fuchsian of the first kind. As a group, $\pi(G)$ is isomorphic to $G$ and is, therefore, finitely generated.
\end{frame}


\begin{frame}
\frametitle{No parabolic elements in the conjugate Fuchsian group}



We claim that the Fuchsian group $\pi(G)$ does not contain parabolic elements. Assume the contrary: let $g\in G$ be such that $\pi(g)$ is parabolic. Hence, there exists a fixed point $w_0\in\mathbb{T}$ of $\pi(g)$ such that
$$(\pi(g))^nw\to w_0,\quad n\to\pm\infty$$
for every $w\in\mathbb{D}.$ Let $w=Z(z),$ $z\in\Lambda(G)_{{\rm int}}$ and let $w_0=Z(z_0),$ $z_0\in\Lambda(G).$ Clearly, 
$$g^n(z)\to z_0,\quad n\to\pm\infty.$$
Hence, $g\in G$ is parabolic, which is not the case (we made an assumption that $G$ does not contain parabolic elements).
\end{frame}

\begin{frame}
\frametitle{Riemann surface of conjugate Fuchsian group is compact}

The assertion below is Theorem 10.4.3 in [Beardon].

\begin{thm} If $\Gamma$ is a finitely generated Fuchsian group of the first kind, then Riemann surface $\mathbb{D}/\Gamma$ has finite area.
\end{thm}

The assertion below is Corollary 4.2.7 in [Katok]. 

\begin{thm} If $\Gamma$ is a Fuchsian group without parabolic elements such that Riemann surface $\mathbb{D}/\Gamma$ has finite area, then $\mathbb{D}/\Gamma$ is compact.
\end{thm}

A combination of these assertions yields that Riemann surface of $\pi(G)$ is compact.
\end{frame}


\begin{frame}
\frametitle{Fundamental domain of conjugate Fuchsian group is compactly supported in $\mathbb{D}.$}

The assertion below is a combination of Corollary 4.2.3 and Theorem 3.2.2 in [Katok].

\begin{thm} If $\Gamma$ is a Fuchsian group whose Riemann surface $\mathbb{D}/\Gamma$ is compact, then there exists a compact fundamental domain $\mathbb{F}$ of $\Gamma.$

\end{thm}

In particular, $\pi(G)$ admits a fundamental domain $\mathbb{F}$ which is compactly supported in $\mathbb{D}.$
\end{frame}

\begin{frame}
\frametitle{The usage of compact fundamental domains}
\begin{lem} We have
$$\sup_{z\in\pi(g)\mathbb{F}}(1-|z|^2)|Z'(z)|\leq\frac{{\rm const}}{|g_{21}|^2}.$$
\end{lem}
\begin{proof} Let $z=\pi(g)w$ with $w\in\mathbb{F}.$ Conformal invariance of hyperbolic metric and the chain rule yield
$$(1-|z|^2)|Z'(z)|=|g'(Z(w))|\cdot(1-|w|^2)|Z'(w)|.$$
Obviously,
$$|g'(Z(w))|=|g_{21}Z(w)+g_{22}|^{-2}=|g_{21}|^{-2}\cdot |Z(w)-g^{-1}(\infty)|^{-2}.$$
The first factor is bounded by $|g_{21}|^{-2}$ and the second one is bounded.
\end{proof}
\end{frame}

\begin{frame}
\frametitle{Critical exponent of the group $G$}
For a Kleinian group $G,$ the series
$$\sum_{g\in G}|g'(z)|^2$$
converges for almost every (with respect to Lebesgue measure) $z\in\bar{\mathbb{C}}.$

The critical exponent of $G$ is defined as follows
$$p=\inf\Big\{q:\ \sum_{g\in G}|g'(z)|^q\mbox{ converges for a.e. }z\in\bar{\mathbb{C}}\Big\}.$$
\end{frame}



\begin{frame}
\frametitle{Proof of the main result, part (a)}
$G$  is a quasiconformal deformation of a Fuchsian group of the first kind. In particular, its limit set $\Lambda(G)$ is a quasi-circle. Hence, the Hausdorff dimension of $\Lambda(G)$ is strictly less than 2.

$G$ is finitely generated and, by the Ahlfors Finiteness Theorem, $G$ is analytically finite. By Bishop-Jones theorem, $G$ is geometrically finite. In particular, its critical exponent $p$ equals to the Hausdorff dimension.

A few hours of meditation over [Sullivan] deliver that $\{\|g\|_{\infty}^{-2}\}_{g\in G}\in l_{p,\infty}.$ Hence, also $\{g_{21}^{-2}\}_{1\neq g\in G}\in l_{p,\infty}.$ By the above lemma, we have that 
$$k\in L_{p,\infty}(\mathbb{D},(1-|z|^2)^{-2}dzd\bar{z}).$$
\end{frame}

\section{Part (b) of the main result}

\begin{frame}
\frametitle{Restatement}
Let $\nu$ be a finite measure such that
$$\varphi(M_{f\circ Z}|[F,M_Z]|^p)=\int_{\Lambda(G)}f(z)d\nu(z).$$
We aim to show that
$$d(\nu\circ g)(z)=|g'(z)|^pd\nu(z).$$
Equivalently, we want 
$$\varphi(M_{f\circ g^{-1}\circ Z}|[F,M_Z]|^p)=\varphi(M_{(f|g'|^p)\circ Z}|[F,M_Z]|^p).$$
\end{frame}

\begin{frame}
\frametitle{Represenation of ${\rm SU}(1,1)$ commutes with $F$}
Let
$$(U_h\xi)(z)=\frac1{\bar{\beta}z+\bar{\alpha}}\xi(\frac{\alpha z+\beta}{\bar{\beta}z+\bar{\alpha}})$$
for every $\xi\in L_2(\mathbb{T})$ and for every $z\in\mathbb{T}.$ Here,
$$h=
\begin{pmatrix}
\alpha&\beta\\
\bar{\beta}&\bar{\alpha}
\end{pmatrix},\quad
|\alpha|^2-|\beta|^2=1.
$$

\begin{lemma} The mapping $h\to U_h$ is a unitary representation of the group ${\rm SU}(1,1)$ which commutes with $F.$
\end{lemma}
\end{frame}

\begin{frame}
\frametitle{Idea of the proof}
We have
$$U_{\pi(g)}\cdot M_{f\circ g^{-1}\circ Z}|[F,M_Z]|^p\cdot U_{\pi(g)}^{-1}=M_{f\circ Z}|[F,M_{g\circ Z}]|^p.$$
Hence,
$$\varphi(M_{f\circ g^{-1}\circ Z}|[F,M_Z]|^p)=\varphi(M_{f\circ Z}|[F,M_{g\circ Z}]|^p).$$

{\bf IF WE HAD}
$$|[F,M_{g\circ Z}]|^p-[F,M_{g\circ Z}]|^p|g'(Z)|^p|\in (\mathcal{L}_{1,\infty})_0,$$
then
$$\varphi(M_{f\circ g^{-1}\circ Z}|[F,M_Z]|^p)=\varphi(M_{(f|g'|^p)\circ Z}|[F,M_Z]|^p).$$
\end{frame}


\begin{frame}
\frametitle{Core lemma}

\begin{lemma} Let $0\leq A\in\mathcal{L}_{\infty}$ and $0\leq B\in\mathcal{L}_{p,\infty}$ be such that $[A^{\frac12},B]\in (\mathcal{L}_{p,\infty})_0,$ then
$$B^pA^p-(A^{\frac12}BA^{\frac12})^p\in (\mathcal{L}_{1,\infty})_0.$$
\end{lemma}

Set $A=M_{|g'|\circ Z}$ and $B=[F,M_{f\circ Z}].$ Long but elementary computation shows that
$$|[F,M_{g\circ Z}]|^p-(A^{\frac12}BA^{\frac12})^p\in (\mathcal{L}_{1,\infty})_0.$$
Applying the lemma, we obtain
$$|[F,M_{g\circ Z}]|^p-B^pA^p\in (\mathcal{L}_{1,\infty})_0.$$
\end{frame}

\section{Proof of the core lemma}

\begin{frame}
\begin{lemma} Let $X,Y\geq0.$ There exists a Schwartz function $g_p$ such that
$$X^p-Y^p=V-\int_{\mathbb{R}}X^{is}VY^{-is}g_p(s)ds,$$
where
$$V=X^{p-1}(X-Y)+(X-Y)Y^{p-1}.$$
\end{lemma}
\begin{proof} It suffices to prove the assertion for the case when $X$ and $Y$ have finite spectra. Multiplying equality on the left by $\chi_{\{\lambda\}}(X)$ and on the right by $\chi_{\{\mu\}}(Y),$ it suffices to prove that
$$\lambda^p-\mu^p=(\lambda-\mu)(\lambda^{p-1}+\mu^{p-1})\cdot\Big(1-\int_{\mathbb{R}}\lambda^{is}\mu^{-is}g_p(s)ds\Big).$$
This is a commutative assertion which can be verified directly.
\end{proof}
\end{frame}

\begin{frame}
\begin{lemma} Let $A,B\geq0.$ We have
$$B^pA^p-Y^p=T(0)-\int_{\mathbb{R}}T(s)g_p(s)ds,$$
where $Y=A^{\frac12}BA^{\frac12}$ and
$$T(s)=B^{p-1+is}[B,A^{p-\frac12+is}]A^{\frac12}Y^{-is}+B^{is}[B,A^{\frac12+is}]A^{\frac12}Y^{p-1-is}.$$
\end{lemma}
Again, it suffices to prove the assertion for the case when $B$ has finite spectrum.
\end{frame}

\begin{frame}
\begin{proof} If $B=\sum_j\lambda_jp_j,$ then
$$B^pA^p-Y^p=\sum_jp_j((\lambda_jA)^p-Y^p).$$
Applying the preceding lemma to $X=\lambda_jA$ and $Y,$ we obtain
$$B^pA^p-Y^p=\sum_jp_j\Big(V_j-\int_{\mathbb{R}}(\lambda_jA)^{is}V_jY^{-is}g_p(s)ds\Big)=$$
$$=\Big(\sum_jp_jV_j\Big)-\int_{\mathbb{R}}\Big(\sum_jp_j(\lambda_jA)^{is}V_jY^{-is}\Big)g_p(s)ds.$$
Here,
$$V_j=(\lambda_jA)^{p-1}(\lambda_jA-Y)+(\lambda_jA-Y)Y^{p-1}.$$
\end{proof}
\end{frame}

\begin{frame}
\begin{proof}
Note that
$$\sum_jp_jV_j=\sum_jp_j(\lambda_j^pA^p-\lambda_j^{p-1}A^{p-1}Y+\lambda_jAY^{p-1}-Y^p)=$$
$$=\Big(\sum_j\lambda_j^pp_j\Big)A^p-\Big(\sum_j\lambda_j^{p-1}p_j\Big)A^{p-1}Y+$$
$$+\Big(\sum_j\lambda_jp_j\Big)AY^{p-1}-\Big(\sum_jp_j\Big)Y^p=$$
$$=B^pA^p-B^{p-1}A^{p-1}Y+BAY^{p-1}-Y^p=T(0).$$
Similarly,
$$\sum_jp_j(\lambda_jA)^{is}V_jY^{-is}=T(s).$$
\end{proof}
\end{frame}

\begin{frame}
\frametitle{Proof of the core lemma I}
We are now ready to prove the core lemma.
\begin{proof} We write
$$B^pA^p-Y^p=B^{p-1}\cdot I+II\cdot Y^{p-1},$$
where
$$I=[B,A^{p-\frac12}]A^{\frac12}+\int_{\mathbb{R}}B^{is}[B,A^{p-\frac12+is}]A^{\frac12}Y^{-is}g_p(s)ds,$$
$$II=[B,A^{\frac12}]A^{\frac12}+\int_{\mathbb{R}}B^{is}[B,A^{\frac12+is}]A^{\frac12}Y^{-is}g_p(s)ds.$$
By H\"older inequality, it suffices to show that $I,II\in(\mathcal{L}_{p,\infty})_0.$
\end{proof}
\end{frame}

\begin{frame}
\frametitle{Proof of the core lemma II}
Consider I. We have
$$[B,A^{p-\frac12+is}]\in (\mathcal{L}_{p,\infty})_0$$
and
$$\|[B,A^{p-\frac12+is}]\|_{p,\infty}\leq(1+|s|)\|A\|_{\infty}^{p-1}\|[B,A^{\frac12}]\|_{p,\infty}.$$

The integrand is measurable in weak operator topology. Since $(\mathcal{L}_{p,\infty})_0$ is a separable Banach space, it follows that integrand is Bochner measurable in $(\mathcal{L}_{p,\infty})_0.$ Since
$$\int_{\mathbb{R}}(1+|s|)|g_p(s)|ds<\infty,$$
it follows that the integrand is Bochner integrable. Hence, $I\in (\mathcal{L}_{p,\infty})_0.$

Similarly, $II\in (\mathcal{L}_{p,\infty})_0.$
\end{frame}





\begin{frame}

{\bf\Huge THANK YOU FOR YOUR ATTENTION}
\end{frame}




\end{document}

